Rapid Prototyping von Strukturen mittels 3D-Druck ist essentiell für die Entwicklung von Mikro- und Nanotechnologie, denn die Einfachheit des Computer-gestützten Designs (CAD) und der additive Herstellungsprozess bergen viele Vorteile im Vergleich zu herkömmlichen Methoden wie Fotolithografie oder Fräsen - beispielsweise die verkürzte Design- und Herstellungszeit, der geringere Materialverbrauch und die Möglichkeit verschiedene Materialien in einem Bauteil zu verwenden.\cite{Vaezi2012,Chu2014}
Durch Forschung an biokompatiblen Materialien und in der Nanopartikeltechnologie, können nun auch elektronische Sensoren und medizinische Prothesen kostengünstig hergestellt werden.\cite{SanchezRomaguera2008,Mohammed2017}
Im Gegensatz zu herkömmlichen Druckern, konnten mit den in diesem Praktikum verwendeten Methoden Leiterbahnen für elektrochemische Sensoren und biokompatible Polymere in Mikrometer-Auflösung gedruckt werden.\cite{Skript}
Ziel des Praktikums war es, das Auflösungsvermögen eines Stereolithografen zu charakterisieren und dessen Prozessprotokoll für den Bau von Flusssensoren in einer Mikrofluidik zu optimieren. Aufgrund von Streueffekten im Harz und an der reflektierenden Oberfläche des Druck-Halters wird vermutet, dass auch an Stellen neben den belichteten Pixel polymerisiert wird. Nach den Drucken sollten somit minimale Öffnungsbreiten oder Durchmesser analysiert werden.
Für Passivierungsdrucke direkt auf Substraten wird getestet, wie optimal diese auf den Substraten bonden und ob sich auf den Öffnungen ein Mikrofilm bildet. Im Falle geschlossener Elektroden soll hier ein Nachätzen der Elektroden mit Schwefelsäure getestet werden. Zudem sollten die verwendeten Sensoren mittels Elektrochemischer Impedanzspektroskopie (EIS) bzw. Cyclovoltammetrie (CV) charakterisiert werden. Ein weiteres Ziel des Praktikum war es, mittels Inkjet-Printing drei-dimensionale Elektrodenstrukturen für den Aufbau eines Zellkultursensor-Systems zu drucken, welches aufgrund technischer Probleme nach zwei Prozessschritten eingestellt wurde.