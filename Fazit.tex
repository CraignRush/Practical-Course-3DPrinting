Im Zuge des Praktikums sollten zwei verschiedene Arbeitspakete bearbeitet werden; zum Einen der \textit{Aufbau eine Zellkultursensor-Systems für die Messung von Aktionspotentialen} und zum Anderen das \textit{Prototyping eines impedanz-basierten Flussratensensors}. Grundlegend lässt sich sagen, dass durch die Neuheit der Technologie und der Geräte eine gewisse Einarbeitungszeit erforderlich ist, bevor effektiv und selbstständig an den Geräte gearbeitet werden kann. Daraus resultiert eine geringere Signifikanz und Genauigkeit der früher erworbenen Daten gegenüber der später Gemessenen. 
Darüber hinaus konnten - bedingt durch Gerätausfälle und Wartezeiten - die \SI{8}{\hour} pro Tag nur bedingt effektiv genutzt werden. Als Konsequenz wurden nicht alle Ziele des Praktikums erreicht. 

Während des Praktikums haben wir Evaluationsstrukturen für den SLA designt, in einem festgelegten Parameterraum gedruckt und analysiert. Somit konnten wir einige Auflösungslimits des Druckers bestimmen bzw. validieren. Des Weiteren konnten wir eine simple Mikrofluidik, sowie Schraubverbindungen für Schlauchfittings 3D-drucken und zusammen mit einer optimalen Passivierungsstruktur mechanisch funktionale Flussratensensoren herstellen. Die Analyse dieser konnte aus Zeitgründen nur teilweise erstellt werden. Das Bonding der Passivierung mit der Mikrofluidik ist immernoch ein sehr unzuverlässig funktionierender Prozess, der weiterer Optimierung bedarf.

Das Zellkultursensor-Projekt wurde nach dem Inkjet-Druck der Elektroden durch einen Defekt am Drucker ausgebremst. So konnten die Silberstrukturen noch mit einem optimierten Galvanisierungsprozess beschichtet werden, dann aber das Proktoll nicht weitergeführt werden. Als Workaround konnten wir die Pillarstrukturen noch mit Aerosol-Jet-Technologie auf die Feedlines drucken. Mangels Zeit konnten aber weder eine 3D-Zellkulturkammer, noch die optimalen Axondioden evaluiert und dementsprechend kein funktionaler Sensor gebaut werden.

Zusammenfassend lässt sich also festhalten, dass zwar nur ein Teil der angepeilten Ziele erreicht wurde, jedoch einige wichtige Erkenntnisse generiert und Prozesse optimiert wurden. Außerdem können auf Basis der erhaltenen Resultate neue Experimente besser geplant bzw. deren Ausgänge besser eingeschätzt werden. Nicht zuletzt ist der Lernerfolg sowohl im Bereich der 3D-Drucktechnologie, als auch der (Elektro-)chemie eine wertvolle Errungenschaft des Praktikums.