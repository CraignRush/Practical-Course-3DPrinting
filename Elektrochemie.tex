\subsection{Elektrochemische Methoden}
Soweit nicht anders spezifiziert, wurden die folgenden Parameter für das jeweilige Messverfahren verwendet. Die generierten Daten wurden mittels der PSTrace 5 (PalmSens) und MATLAB (MathWorks) Software analysiert und geplottet.
\subsubsection{Elektrochemische Impedanzspektroskopie}
Um die Durchlässigkeit des Passivierungslayers bzw. die Impedanz der einzelnen Elektroden zu bestimmen, wurde ein Impedanzspektrogramm mit einem Multiplex - Potentiostaten (Palmsens 3, PalmSens) erstellt. In diesem wurde in einem Bereich von \SIrange{0.1}{1e6}{Hz} mit einem Gleichanteil von \SI{0}{\volt}, sowie einem Wechselanteil von \SI{100}{\milli \volt} der resultierende Strom gegen eine Ag/AgCl Elektrode in einer Umgebungslösung von \SI{5}{\milli\Molar} Kaliumhexacyanidoferrat(III) (FeCy) in \SI{100}{\milli\Molar} Kaliumchlorid (KCl) Lösung in 41 Schritten gescannt. Für diese Messung wurde ein Sensorchip mit der unpassivierten Seite in einen Sockel (IC51-390, Yamaichi) eingespannt. Anschließend wurden \SI{500}{\micro\liter} der Standardlösung einpipettiert und die Referenz- bzw. Arbeitselektrode darin positioniert.

\subsubsection{Cyclovoltammetrie}
Die Messungen der CV gliederten sich in zwei Unterabschnitte. Zum Einen wurde die Sensitivität der Elektroden mittels CV vermessen, zum Anderen wurde CV genutzt, um während der Elektrodenreinigung mit Schwefelsäure eine Oxidation der Elektrodenoberfläche um einige Atomlagen zu erzeugen. Generell wurde die CV in einer Drei-Elektroden-Konfiguration gegen ein Ag/AgCl-Elektrode aufgenommen.
\paragraph{Sensitivitätscharakterisierung}
Für eine Sensitivitätscharakterisierung wurden zwei CV in \SI{5}{\milli\Molar} FeCy in \SI{100}{\milli\Molar} KCl-Lösung von \SIrange{-0.2}{0.6}{\volt} in \SI{10}{\milli\volt} Schritten und mit einer Scan-Rate von \SI{50}{\milli\volt\per\second} aufgenommen.
\paragraph{Reinigungs-CV}
Zur Reinigung der Elektroden in Schwefelsäure wurden diese (wiederum durch den Yamaichi-Sockel an den Potentiostaten angeschlossen) mit ca \SI{300}{\micro\liter} \SI{13}{\percent} H$_2$SO$_4$ (Schwefelsäure, Sigma) inkubiert und kurzgeschlossen. Während der Inkubation erzeugte man 40 CV-Hysteresen an allen (kurzgeschlossenen) Elektroden von \SIrange{-0.5}{1.5}{\volt} in \SI{10}{\milli\volt} Schritten mit einer Scan-Rate von \SI{500}{\milli\volt\per\second}. Schlussendlich wurde die Schwefelsäure mehrmals mit dH$_2$O von den Elektroden gewaschen.