Die in einem Praktikum generierten Ergebnisse bergen einige Unsicherheiten, die sich durch fehlende Gerätekenntnis und mangelnde Erfahrung in speziellen Forschungsgebieten ergeben. Es wurde versucht den menschlichen Fehler durch die Etablierung von Standardverfahren und -prozessen zu minimieren, jedoch lässt sich nicht ausschließen, dass bei manuellen Prozessen einige systematische Fehler gemacht wurden.

Es ist beispielsweise immer noch fraglich, wie sich bei einer geringeren Leistung des SLA eine geringere Tiefe des Kanals - also mehr polymerisiertes Material im Kanal - ergeben kann. Auch die Ergebnisse zum Bonding können durch eine schlechte Silanisierung oder Fehler in einem der anderen Prozesschritte erklärt werden, zumal der Prozess vor dem Praktikum als etabliert galt. 

Aufgrund des kleinen Zeitrahmens konnten auch einige Daten, die durch die Oberflächenvermessung der Teststrukturen zur Druckerauflösung bereits generiert wurden, nicht analysiert werden. Hier sind beispielsweise die um \SI{45}{\degree} gedrehten Kanäle bzw. Wells und die Axondioden zu nennen.

Darüber hinaus steht auch eine definitive Aussage über die Höhe des Mikrofilms auf den Sensorelektroden aus. Die hier verwendeten Messverfahren konnten den Sachverhalt nicht abschließend klären. Als möglicher Beweis könnte das Abkleben der Hälfte eines Glaschips vor der Passivierung dienen. Dadurch könnte man auf einer Seite den Abstand zwischen Passivierung und unbehandelten Elektroden messen, während auf der anderen Seite die üblichen Öffnungen in der Passivierung vermessen werden. Eine andere Möglichkeit könnte eine optische Profilometrie auf Basis der unterschiedlichen Grenzflächeneigenschaften von PEGDA-Glas und Luft-Glas bzw. PEGDA-Gold und Luft-Gold sein.

