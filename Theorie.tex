\subsection{Stereolithographie}

Stereolithographisches (SLA) 3D-Drucken kreiert feste dreidimensionale Objekte Schicht für Schicht mittels additiver Fertigungstechnik. Dabei wird ein Halter, auf welchem die Struktur gedruckt wird, in ein mit Harz gefülltes Reservoir eingetaucht und Schicht für Schicht angehoben. Anfangs taucht der Halter bis zum Boden, hebt sich dann um eine Schichthöhe und nun wird die erste Schicht belichtet, wobei die Belichtung das Harz polymerisiert und aushärtet. Für die nächste Schicht fährt der Halter wieder eine Schicht höher und die nächste wird belichtet, welche auch mit der ersten Schicht polymerisiert und so tatsächlich auch eine chemische Verbindung hergestellt wird, anders zu sog. Fused-Deposition-Modeling (FDM) Drucker, die die einzelnen Schichten nur physikalisch bzw. thermisch miteinander verbinden.

\subsection{Tintenstrahldrucken}

Ein Tintenstrahl oder Inkjet Drucker druckt sowohl zweidimensionale als auch dreidimensionale Strukturen, jedoch in kleineren Auflösungen verglichen mit dem SL-Drucker. Zudem können hier mit unterschiedlichen Tinten gedruckt werden, die beispielsweise leitfähig oder wasserlöslich sein können. Die Flüssigkeit im Tintenreservoir wird mit Hilfe eines piezoelektrischen Wandlers angeregt, sodass aus einer sehr schmalen Düse einzelne Tropfen mit Volumen von wenigen \si{\pico\liter} herausschießen, auf dem Substrat auftreffen und dort Punkte bilden. Mit geschickter Ansteuerung des Druckkopfs und Überlagerung der einzelnen Punkte können dreidimensionale Strukturen erzeugt werden.
